\begin{abstract}
AI is undergoing a paradigm shift with the rise of models (\eg~BERT, DALL-E, GPT-3)
that are trained on broad data at scale and are adaptable to a wide range of downstream tasks.
We call these models \emph{foundation models} to underscore their critically central yet incomplete character.
This report provides a thorough account of the opportunities and risks of foundation models, ranging from their capabilities (\eg~language, vision, robotics, reasoning, human interaction) and technical principles (\eg~model architectures, training procedures, data, systems, security, evaluation, theory) to their applications (\eg~law, healthcare, education) and societal impact (\eg~inequity, misuse, economic and environmental impact, legal and ethical considerations).
Though foundation models are based on standard deep learning and transfer learning,
their scale results in new emergent capabilities,
and their effectiveness across so many tasks incentivizes homogenization.
Homogenization provides powerful leverage but demands caution,
as the defects of the foundation model are inherited by all the adapted models downstream.
%We emphasize that this report is not an unequivocal endorsement of foundation models, but rather a measured assessment:
Despite the impending widespread deployment of foundation models,
we currently lack a clear understanding of how they work, when they fail,
and what they are even capable of due to their emergent properties.
To tackle these questions,
%Further, they might pose societal risks, by exacerbating inequity, centralizing power, and amplifying disinformation.
%Through our own efforts in preparing this report,
we believe much of the critical research on foundation models
will require deep interdisciplinary collaboration commensurate with their fundamentally sociotechnical nature.
\end{abstract}